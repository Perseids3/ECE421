\documentclass{article}
\usepackage{color}
\usepackage{graphicx}
\usepackage{setspace}
\usepackage{hyperref}
\usepackage{geometry}
\usepackage{amsmath}
\usepackage{enumitem,amssymb}
\usepackage{pifont}
\definecolor{darkgreen}{rgb}{0.0, 0.5, 0.0}
\geometry{left = 1.25in, right=1.25in} % New Stuff Learned!
\newcommand{\cmark}{\ding{51}}
\newcommand{\xmark}{\ding{55}}
\newcommand{\done}{\rlap{$\square$}{\raisebox{2pt}{\large\hspace{1pt}\cmark}}
\hspace{-2.5pt}}
\newcommand{\wontfix}{\rlap{$\square$}{\large\hspace{1pt}\xmark}}
\newlist{todolist}{itemize}{2}
\setlist[todolist]{label=$\square$}
\doublespacing
\begin{document}
\begin{titlepage}

\title{\textbf{ECE421 Week 1}}
\author{\textit{Sanzhe Feng}}
\date{\textit{\today}}
\maketitle
\end{titlepage}
\setlength{\parindent}{0pt}

\subsection*{ECE421 Course Info}
\url{https://docs.google.com/spreadsheets/d/e/2PACX-1vRUK3Fm0IFLHNxh1yKRssrEUolh-COzyp-3zGsovXtdKv_cQR90TpA91kpnkGCS6wzxgqjYWkL5aI1f/pubhtml}

MAKE SURE TO USE OFFICE HOURS!!

\begin{todolist} %[\done]
    \item \textbf{FRIDAY NIGHT Participation Form}
    \item \textbf{WEEKEND Asyn Videos}
    \item \textbf{MONDAY 1PM Vote for questions}
    \item \textbf{TUESDAY Lecture}
    \item \textbf{THURSDAY NIGHT Assignment}   
\end{todolist}

Assignment INFO: \textbf{BOTH CROWDMARK AND QUERCUS}


\subsection*{What is Machine Learning}
Programs a single algorithm that learns from data:

$def \hspace{0.5cm}learning\_algorithm (X, Y)$ 

\hspace{1.1cm}$return algorithm $

\hspace{1.1cm}$algorith(x) = y$



\subsection*{Why use machine learning}
For many problems, its hard to program the correct behavior by hand: facial and speech rocogonization.
Also, the algorithm may need to change throughout time (adapt). And we might believe such algorithm that performs better than human.

\subsection*{Types of Machine Learning}

1. Supervised learning: with labeled input and output. The goal is to predict correct label.

2. Reinforcement learning: reward signal. Algorithm detect how to maximize reward signal when performing a task.

3. Unsupervised learning: unlabled data. Typically used for looking for special patterns in data.


\subsection*{A typical ML pipeline}

1. Input representation: what each dimension of x contains

2. Model hypothesis class: $y = g (wx+b)$

3. Training algorithm to find parameters ($w$ and $b$)

4. Test this model

\end{document}